\documentclass{article}

\title{Formula Readme}
\date{\today}
\author{Paul Jones\\ Computer Architecture (01:198:211) \\ School of Arts and Sciences \\ Rutgers University}

\begin{document}

\maketitle

% this file should describe your design and implementation of the formula program. 

% In particular, it should detail your design, any design/implementation chal- lenges that you ran into, 

\section{Challenges}

Figuring out how to make a C file ``talk to'' an Assembly file was the biggest challenge I ran into.
The linking process is made clear when you visualize what it is that is actually happening when 
you compile a C file on a particular machine, but from a programmer's perspective this does not
trivially follow.

Conceiving the level on which software finally meets hardware is still hugely difficult to me.
It is not \emph{a priori} obvious to me that software actually can meet hardware.
I am cognizant of the fact that it does, and all the time, otherwise running a program would be impossible,
including this operating system and the programs it runs for me.
Yet my confusion remains.

The difficulty I have with the point which software becomes hardware is analogous to the mind-body distinction.
I am aware that I have a physical brain, yet my mind seems distinct from it.
Consciousness seem distinct from the brain the way that software seems distinct from hardware.

%and an analysis (e.g., big-O analysis) of the space and time performance of your program.

\section{Big-O Analysis}

\subsection{Space}

My implementation only ever requires a 32-bit integer at any given time.
Furthermore, my assembly is written for a 32-bit processor, so any value that cannot be stored
in a 32-bit binary integer will flag overflow and return nothing.

This means that the maximum value one can input is 12, as 13 will cause overflow on the \texttt{mull}
Assembly command.

\subsection{Time}

It is amazing that the most ``expensive'' thing that my program does is print to the terminal.
The comparisons are done on an incredibly low level, and the process that takes the most time
is printing each integer.

\end{document}